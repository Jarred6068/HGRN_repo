\documentclass[a4paper,12pt]{article}
\usepackage [left=25.4mm,top=25.4mm]{geometry}
\usepackage{amsmath}
\usepackage{amssymb}
\usepackage{graphicx}
%\usepackage{apacite}
\usepackage{url}
\usepackage{subfig}
\usepackage{csvsimple}
\usepackage{float}
\usepackage{lineno}
\usepackage[affil-it]{authblk}
\usepackage{setspace}
\usepackage{makecell} 
\usepackage{tikz}
\usepackage{csvsimple}
\usepackage{newfloat}
\usepackage{xcolor}
\usepackage{tabularx,booktabs}
\usepackage{multirow}
\usepackage{multicol}
\usepackage{array}
\usepackage{tocbasic}
\usepackage{sectsty}

\sectionfont{\fontsize{12}{15}\selectfont}
\chapterfont{\fontsize{14}{15}\selectfont}
\subsectionfont{\fontsize{10}{15}\selectfont}

\newcommand\hcancel[2][black]{\setbox0=\hbox{$#2$}%
	\rlap{\raisebox{.45\ht0}{\textcolor{#1}{\rule{\wd0}{1pt}}}}#2} 
\newcommand{\forceindent}{\leavevmode{\parindent=2em\indent}}

\DeclareFloatingEnvironment[name={Supplementary Figure},fileext=lsf,listname={List of Supplementary Figures}]{suppfigure}
\DeclareFloatingEnvironment[name={Supplementary Table},fileext=lsf,listname={List of Supplementary Tables}]{supptable}
\DeclareFloatingEnvironment[name={Supplementary Material},fileext=lsf,listname={List of Supplementary Material}]{suppmat}

\newcolumntype{P}[1]{>{\centering\arraybackslash}p{#1}}
\newcolumntype{M}[1]{>{\centering\arraybackslash}m{#1}}

\DeclareMathOperator*{\argmax}{arg\,max}
\DeclareMathOperator*{\argmin}{arg\,min}

\newcommand{\indep}{\perp \!\!\! \perp}
\renewcommand*\contentsname{TABLE OF CONTENTS}
\renewcommand{\listfigurename}{LIST OF FIGURES}
\renewcommand{\listtablename}{LIST OF TABLES}

\renewcommand{\arraystretch}{1.5}


\begin{document}
	
	\begin{titlepage}
		\title{Data Simulation Algorithm For Hierarchical Networks}
		\author[1]{Jarred M. Kvamme}
		\author[2]{Boyu Zhang}
		\author[1,3]{Audrey Q. Fu}
		\affil[1]{Department of Bioinformatics and Computational Biology - University of Idaho}
		\affil[2]{Department of Computer Science - University of Idaho}
		\affil[3]{Department of Mathematics and Statistical Science - University of Idaho}
		\maketitle
	\end{titlepage}
	
	
	\newpage
	\tableofcontents{}
	\addcontentsline{toc}{section}{LIST OF TABLES}
	\addcontentsline{toc}{section}{LIST OF FIGURES}
	\listoftables
	\listoffigures
	\newpage
	
	
	
	\section*{Simulating The Network}
	
	\begin{itemize}
		\item [\bf 1. ] Generate the top layer Graph:
		
			\begin{itemize}
				\item [1.1] initiate the graph of the top layer (i.e $\ell^{th}$ layer) consisting of $n_\ell$ nodes.  
				\item [\bf If] the top layer is fully connected:
				\begin{itemize}
					\item [1.1.1] Generate a Wattz Strogratz graph with $n_\ell$ nodes. Each node is connected to its $n_\ell$ nearest neighbors and with probability $p_\ell$ of rewiring. This yields a graph $\mathcal{G}_\ell = \{E,V\}$ with edges $E =\{ {e_{ij}}\ \forall \ i \neq j\}$ and nodes $V = \{n_i\}_{i =1}^{n_\ell}$
				\end{itemize}
				\item [\bf If] the top layer is disconnected:
				\begin{itemize}
					\item [1.1.2] Generate a graph of $n_\ell$ nodes with no edges. This yields a graph $\mathcal{G}_\ell = \{E, V\}$ where $E = \emptyset$ and $V = \{n_i\}_{i =1}^{n_\ell}$
				\end{itemize}
			\end{itemize}
			
		\item [\bf 2.] For the remaining $\ell - 1$ layers in the hierarchy:
		
		\begin{itemize}
			\item [] For each node in $\mathcal{G}_\ell$:
			\begin{itemize}
				\item [2.1] Generate a small world, scale free, or random subgraph $\mathcal{G}_{\ell - 1}$ with $m \sim \text{Uniform}(a,b)$ nodes:
				\item [\bf If] the subgraph is a small world graph:
				\begin{itemize}
					\item [] $\mathcal{G}_{\ell - 1}$ is generated as a Watts Strogatz graph with $m$ nodes where each node is connected to its $k$ nearest neighbors and with probability of rewiring $p_{\ell - 1} = p_\ell / (\frac{1}{2}a+b)$  
				\end{itemize}
				\item [\bf If] the subgraph is a random graph:
				\begin{itemize}
					\item [] $\mathcal{G}_{\ell - 1}$ is generated as a Erdős-Rényi graph with $m$ nodes and $E\sim \text{uniform}(a,b)$ edges
				\end{itemize}
				\item [\bf If] the subgraph is a scale free graph:
				\begin{itemize}
					\item [] $\mathcal{G}_{\ell - 1}$ is generated as a Barabási-Albert graph with $m$ nodes and $g \sim \text{uniform}(2, m-1)$ possible edges for each node
				\end{itemize}
			\end{itemize}
		\end{itemize}
		
	\end{itemize}
	
	
	
	
	
	
	
	
	
	
	
	
	
	
	
	

	
	
	
	
	
	
	
	
	
	
\end{document}